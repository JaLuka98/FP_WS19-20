\section{Diskussion}
\label{sec:Diskussion}

Insgesamt ist der Versuch als (MAL SEHEN) zu bewerten.

Im ersten Teil der Auswertung konnte gezeigt werden, dass die Heizraten mit
durchschnittlich
\begin{align*}
  b_1&=\SI{1.1339(0024)}{\kelvin\per\minute}= \SI{0.01890(00004)}{\kelvin\per\second}\,, \\
  b_2&=\SI{1.7587(0033)}{\kelvin\per\minute}= \SI{0.02931(00005)}{\kelvin\per\second}\,
\end{align*}
recht konstant gehalten werden konnten, auch wenn sie nicht ganz bei den angestrebten
Werten liegen. An den geringen Unsicherheiten ist zu erkennen, dass die Werte gut auf einer
Geraden liegen.

Der Untergrund lässt sich gut mit einer Exponentialfunktion an die Daten anpassen.
Hier ist anzumerken, dass nur wenige Punkte für die Ausgleichsrechnung verwendet
werden konnten, sodass die Parameter der Ausgleichsrechnung große Unsicherheiten
aufweisen. Außerdem wurden die Werte, die in der Ausgleichsrechnung berücksichtigt wurden,
alle per Hand ausgewählt. Dennoch ist in den Abbildungen \ref{fig:depol1} und \ref{fig:depol2}
zu erkennen, dass die Ausgleichsrechnungen gut zu den Daten passen.

Die Bestimmung der Aktivierungsenergie über die Stromdichte ist stark zu kritisieren.
Hier sollte theoretisch ein annähernd linearer Verlauf vorliegen. Dieser ist jedoch
in den Abbildungen \ref{fig:fit1} und \ref{fig:fit2} nur in Ansätzen erkennbar.
Entsprechend erscheint es auch nicht sinnvoll hier eine lineare Ausgleichsrechnung
durchzuführen. Die Ausgleichsrechnung kann daher nur für wenige Werte durchgeführt
werden, wobei durch die Auswahl der verwendeten Daten auch das Ergebnis stark beeinflusst
wird. Die so bestimmten Aktivierungsenergien liegen bei
\begin{align}
  WERTE
\end{align}
und weichen damit um (WERTE) vom Theoriewert ab.
