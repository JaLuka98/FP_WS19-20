\newpage
\section{Diskussion}
\label{sec:Diskussion}

Insgesamt ist der Versuch nur bedingt als erfolgreich zu bewerten.

Im ersten Teil der Auswertung konnte gezeigt werden, dass die Heizraten mit
durchschnittlich
\begin{align*}
  b_1&=\SI{1.1339(0024)}{\kelvin\per\minute}= \SI{0.01890(00004)}{\kelvin\per\second}\,, \\
  b_2&=\SI{1.7587(0033)}{\kelvin\per\minute}= \SI{0.02931(00005)}{\kelvin\per\second}\,
\end{align*}
recht konstant gehalten werden konnten, auch wenn sie nicht ganz bei den angestrebten
Werten liegen. An den geringen Unsicherheiten ist zu erkennen, dass die Werte gut auf einer
Geraden liegen.

An dem Verlauf der Kurven ist zu erkennen, dass das Maximum für den Depolarisationsstrom,
wie es auch theoretisch zu erwarten ist, für die Messung mit der geringeren Heizrate bei
einer niedrigeren Temperatur vorliegt als bei der Messung mit der höheren Heizrate. Das ist
sinnvoll, da bei einer geringeren Heizrate schon früher mehr Dipole relaxieren konnten, sodass
das Maximum früher erreicht wird. Außerdem ist das Maximum hier entsprechend niedriger als
bei der Messung mit der höheren Heirate.

Der Untergrund lässt sich gut mit einer Exponentialfunktion an die Daten anpassen.
Hier ist anzumerken, dass nur wenige Punkte für die Ausgleichsrechnung verwendet
werden konnten, sodass die Parameter der Ausgleichsrechnung große Unsicherheiten
aufweisen. Außerdem wurden die Werte, die in der Ausgleichsrechnung berücksichtigt wurden,
alle per Hand ausgewählt, was eine weitere Fehlerquelle darstellt. Dennoch ist in den
Abbildungen \ref{fig:depol1} und \ref{fig:depol2} zu erkennen, dass die Ausgleichsrechnungen
gut zu den Daten passen.

Die Vorgehensweise bei der Bestimmung der Aktivierungsenergie über die Stromdichte ist zu kritisieren.
Hier sollte theoretisch ein annähernd linearer Verlauf vorliegen. Dieser ist jedoch
in den Abbildungen \ref{fig:fit1} und \ref{fig:fit2} nur in Ansätzen erkennbar.
Entsprechend erscheint es auch nur bedingt sinnvoll hier eine lineare Ausgleichsrechnung
durchzuführen. Die Ausgleichsrechnung kann daher nur für wenige Werte durchgeführt
werden, wobei durch die Auswahl der verwendeten Daten das Ergebnis beeinflusst
wird. Die so bestimmten Aktivierungsenergien liegen bei
\begin{align}
  W_1&=\SI{1.04(06)e-19}{\joule}= \SI{0.652(034)}{\eV}  \,, \\
  W_2&=\SI{1.24(10)e-19}{\joule}=\SI{0.77(06)}{\eV} \,.
\end{align}
und weichen damit um $-1{,}21\%$ für $W_1$ und $16{,}67\%$ für
$W_2$ vom Literaturwert $0{,}66\,$eV ab. Die Methode liefert also trotz einiger
Ungenauigkeiten sehr gute Ergebnisse.
Gründe dafür, dass die Daten nicht-linear sind könnten darin liegen, dass der Untergrund
nicht richtig bestimmt werden konnte, oder dass die Temperatur schon zu hoch war und somit
die lineare Näherung in diesem Temperaturbereich keine Gültigkeit mehr hat.

Die Bestimmung der Aktivierungsenergie aus dem Polarisationsanstaz bringt ebenfalls
einige Ungenauigkeiten mit sich, weil die oberen Grenzen für die Integration per Hand
ausgewählt werden müssen. Zudem wurden gezielt Messwerte nicht berücksichtigt, was
dazu führt, dass die Ergebnisse an Aussagekraft verlieren. Die Aktivierungsenergien
können so zu
\begin{align*}
	W_1=\SI{1.40(4)e-19}{\joule}=\SI{0.876(23)}{\eV} \,, \\
	W_2=\SI{1.51(4)e-19}{\joule}=\SI{0.940(26)}{\eV} \,
\end{align*}
bestimmt werden. Die Abweichungen vom oben genannten Theoriewert betragen hier
$32{,}72\%$ für $W_1$ und $42{,}42\%$ für $W_2$.

Die Relaxationszeiten wurden zu
\begin{align*}
	\tau_{0,1}=\SI{1.9(20)e-15}{\second} \,, \\
	\tau_{0,2}=\SI{1.2(14)e-16}{\second} \,
\end{align*}
bestimmt. Die Werte enthalten keine Information, da hier die Unsicherheiten größer
als die Nominalwerte sind. Außerdem weichen sie stark vom Literaturwert  $\SI{4(2)e-14}{\second}$
ab.
