\newpage
\section{Diskussion}
\label{sec:Diskussion}

Insgesamt ist der Versuch nur bedingt als erfolgreich zu bewerten.

Im ersten Teil der Auswertung konnte gezeigt werden, dass die Heizraten mit
durchschnittlich
\begin{align*}
  b_1&=\SI{1.1339(0024)}{\kelvin\per\minute}= \SI{0.01890(00004)}{\kelvin\per\second}\,, \\
  b_2&=\SI{1.7587(0033)}{\kelvin\per\minute}= \SI{0.02931(00005)}{\kelvin\per\second}\,
\end{align*}
recht konstant gehalten werden konnten, auch wenn sie nicht ganz bei den angestrebten
Werten liegen. An den geringen Unsicherheiten ist zu erkennen, dass die Werte gut auf einer
Geraden liegen.

An dem Verlauf der Kurven ist zu erkennen, dass das Maximum für den Depolarisationsstrom,
wie es auch theoretisch zu erwarten ist, für die Messung mit der geringeren Heizrate bei
einer niedrigeren Temperatur vorliegt als bei der Messung mit der höheren Heizrate. Das ist
sinnvoll, da bei einer geringeren Heizrate schon früher mehr Dipole relaxieren konnten, sodass
das Maximum früher erreicht wird. Außerdem ist das Maximum hier entsprechend niedriger als
bei der Messung mit der höheren Heirate.

Bei der Messung mit der höheren Heizrate ist am Anfang ein kleines Maximum zu
beobachten. Dieses könnte durch Störungen des Messgeräts durch elektromagnetische Felder
entstanden sein. Da im weiteren Verlauf der Kurve jedoch kein weiter solcher Peak
zu sehen ist, erscheint diese Erklärung nicht sehr wahrscheinlich. Da der Peak zu den
ersten aufgenommenen Messdaten zählt, könnte auch eine nicht korrekte Durchführung des
Experiments der Grund für den Peak sein.

Der Untergrund lässt sich gut mit einer Exponentialfunktion an die Daten anpassen.
Es ist anzumerken, dass der Wertebereich der in der Ausgleichsrechnung berücksichtigt wurde,
per Augenmaß bestimmt wurde. In den
Abbildungen \ref{fig:depol1} und \ref{fig:depol2} ist aber zu erkennen, dass die Ausgleichsrechnungen
gut zu den Daten passen, wodurch die verwendeten Wertebereiche als sinnvoll bewertet werden können.

Die Vorgehensweise bei der Bestimmung der Aktivierungsenergie über die Stromdichte ist zu kritisieren.
Hier sollte theoretisch ein annähernd linearer Verlauf vorliegen. Dieser ist jedoch
in den Abbildungen \ref{fig:fit1} und \ref{fig:fit2} nur in Ansätzen erkennbar.
Entsprechend erscheint es auch nur bedingt sinnvoll hier eine lineare Ausgleichsrechnung
durchzuführen. Die Ausgleichsrechnung kann daher nur für wenige Werte durchgeführt
werden, wobei durch die Auswahl der verwendeten Daten das Ergebnis beeinflusst
wird. Die so bestimmten Aktivierungsenergien liegen bei
\begin{align*}
 W_1&=\SI{1.01(05)e-19}{\joule}= \SI{0.633(033)}{\eV}  \,, \\
 W_2&=\SI{1.22(010)e-19}{\joule}=\SI{0.76(06)}{\eV} \,.
\end{align*}
Der Literaturwert hierzu beträgt $W_{\text{lit}}=0{,}66\,$eV \cite{lit}. Davon weichen
die aus den Messwerten berechneten Werte um $-4{,}10\%$ für $W_1$ und $15{,}07\%$ für
$W_2$ ab.
Die Methode liefert also trotz einiger
Ungenauigkeiten sehr gute Ergebnisse.
Gründe dafür, dass die Daten nicht-linear sind, könnten sein, dass der Untergrund
nicht richtig bestimmt werden konnte, oder dass die Temperatur schon zu hoch war und somit
die lineare Näherung in diesem Temperaturbereich keine Gültigkeit mehr hat.

Die Bestimmung der Aktivierungsenergie aus dem Polarisationsansatz bringt ebenfalls
einige Ungenauigkeiten mit sich, weil die oberen Grenzen für die Integration per Hand
ausgewählt werden müssen. Zudem wurden gezielt Messwerte nicht berücksichtigt, was
dazu führt, dass die Ergebnisse an Aussagekraft verlieren. Die Aktivierungsenergien
können so zu
\begin{align*}
	W_1=\SI{1.373(033)e-19}{\joule}=\SI{0.857(021)}{\eV} \,, \\
	W_2=\SI{1.51(04)e-19}{\joule}=\SI{0.943(026)}{\eV} \,
\end{align*}
bestimmt werden.
Die Abweichungen vom oben genannten Theoriewert betragen hier
$29{,}88\%$ für $W_1$ und $42{,}95\%$ für $W_2$. Diese Abweichung sind deutlich größer als bei der ersten Methode, weswegen diese geeigneter erscheint.

Die Relaxationszeiten wurden zu
\begin{align*}
	\tau_{0,1}=\SI{4.62(443)e-15}{\second} \,, \\
	\tau_{0,2}=\SI{1.03(121)e-16}{\second} \,
\end{align*}
bestimmt.
Die Werte sind nicht allzu weit von dem Literaturwert von $\SI{4(2)e-14}{\second}$ \cite{lit} entfernt, der ebenso eine hohe Ungenauigkeit aufweist, wenn berücksichtigt wird, dass die Bestimmung von $\tau_0$ sehr sensitiv für Abweichungen in $W$ ist, wie bereits in Kapitel \ref{subsec:relax} erläutert.

Deswegen erübrigt sich auch ein Vergleich der beiden Methoden bezüglich der Relaxationszeit, da sich um Größenordnungen verschiedene Werte ergeben. Um eine aussagekräftige Relaxationszeit zu ermitteln, dessen Fehler deutlich geringer als der Nominalwert ist, müsste die Aktivierungsenergie viel genauer bestimmt werden. Es wurde mit den vorliegenden Messdaten validiert, dass das Herauslassen auch nur eines Punktes in der Ausgleichsrechnung für den Polarisationsansatz zu mehr als doppelt so großen oder kleinen Relaxationszeiten führen kann, sodass die Bestimmung der Relaxationszeit insgesamt als sehr instabil zu bewerten ist. Dies deckt sich mit der großen Unsicherheit des Literaturwerts.
