\section{Theorie}
\label{sec:Theorie}

Ein Ionenkristall ist eine räumlich periodische Anordnung aus Anionen und Kationen, die durch ionische Bindungen zusammengehalten werden. Bei Caesiumiodid (CsI) sind die Caesiumionen \ce{Cs^+} Kationen, während die Iodidionen \ce{I^+} die Anionen sind.
Durch die Dotierung von solchen Alkalimetallsalzen mit zweiwertigen Kationen wie zum Beispiel Stromtiumionen \ce{Sr^{2+}} können permanente elektrische Dipole erzeugt werden, da der gesamte Kristall ladungsneutral bleiben muss und sich eine Leerstelle bildet. Dies ist in Abbildung \ref{fig:dipolIonenkristall} zu sehen.

Es ist auch ersichtlich, dass es nur diskrete Ausrichtungen des Dipols $\vec{P}$ im Festkörper gibt, da dieser vollständig durch die Lage von Leerstelle und Fremdion bestimmt ist. Durch Diffusion der Leerstelle kann der Dipol seine Orientierung ändern. Dazu ist eine materialspezifische Aktivierungsenergie $W$ nötig, da das gitterperiodische Potenzial überwunden werden muss. Die mittlere Zeit zwischen zwei Umorientierungen des Dipols wird Relaxationszeit $\tau$ gennant und beträgt
\begin{equation}
  \tau(T) = \exp(\frac{W}{k_{\text{B}}T})
  \label{eqn:relaxtime}
\end{equation}
mit der Temperatur des Kristalls $T$ und der Boltzmannkonstante $k_{\text{B}}$.
