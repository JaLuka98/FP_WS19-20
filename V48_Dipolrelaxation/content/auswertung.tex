\section{Auswertung}
\label{sec:Auswertung}

\subsection{Bestimmung der tatsächlichen Heizraten}
Zunächst sollen die tatsächlichen Heizraten bestimmt werden. Die angestrebten
Heizraten waren $1{,}2\,$K/min und 2\,K/min. Die tatsächlichen Heizraten lassen
sich mithilfe einer Ausgleichsrechnung der Form
\begin{equation*}
  f(T)=aT+n
\end{equation*}
berechnen. Dabei wird der bei der niedrigstn Temperatur gemessene Wert als bekannter
Punkt festgelegt. Die Heizrate ergibt sich aus der Steigung der Geraden. Sie beträgt
\begin{align*}
  b_1&=\SI{1.1339(0024)}{\Kelvin \per \minute}= \,, \\
  b_2&=\SI{1.7587(0033)}{\Kelvin \per \minute}= \,.
\end{align*}
Die Messwerte und die Ausgleichsfunktionen sind in den Abbildungen \ref{fig:heiz1}
und \ref{fig:heiz2} zu sehen.

\begin{figure}
  \centering
  \includegraphics[width=\textwidth]{build/interpol_1_2.pdf}
  \caption{Messwerte und lineare Ausgleichsrechnung zur Bestimmung der tatsächlichen
  Heizrate für die Messreihe mit niedrigerer Heizrate.}
  \label{fig:heiz1}
\end{figure}
\begin{figure}
  \centering
  \includegraphics[width=\textwidth]{build/interpol_2.pdf}
  \caption{Messwerte und lineare Ausgleichsrechnung zur Bestimmung der tatsächlichen
  Heizrate für die Messreihe mit höherer Heizrate.}
  \label{fig:heiz2}
\end{figure}


\subsection{Bestimmung und Abzug des Untergrundes}

Zur Bestimmung des Untergrundes müssen zunächst die Minima des Stroms gefunden
werden. Für die gefundenen Werte wird anschließend eine Ausgleichsrechnung der
Form
\begin{equation*}
  f(T)=a e^{bT} +c
\end{equation*}
durchgeführt. Es ergeben sich die Parameter
\begin{align*}
  WERTE EINFUEGEN!
\end{align*}

Die Ausgleichsfunktionen für den Untergrund, sowie die für die Ausgleichsrechnungen
verwendeten Werte sind in den Abbildungen (REFERENZ) und (REFERENZ) zu sehen.


Der Untergrund wird anschließend von den Daten subtrahiert.

\subsection{Bestimung der Aktivierungsenergie über die Stromdichte}

\subsection{Bestimmung der Aktivierungsenergie aus dem Polarisationsansatz}
