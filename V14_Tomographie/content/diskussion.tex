\section{Diskussion}
\label{sec:Diskussion}

Zusammenfassend lässt sich der Versuch nur bedingt als erfolgreich bewerten.

Im Spektrum ist ein eindeutiger Peak Peak zu sehen. Außerdem ist ein Compton-Spektrum
zu sehen, das aus der Wechselwirkung der Gammaphotonen mit dem Szintillator entsteht.

Die gemessenen Zählraten für die Nullmessung mit einem hohlen Aluminiummantel
ergeben sich zu
\begin{align*}
  I_0,1&=\SI{171.83(135)}{\per\second} \,, \\
  I_0,2&=\SI{169.63(133)}{\per\second} \,, \\
  I_0,3&=\SI{158.78(133)}{\per\second} \,.
\end{align*}

Für die Absorptionskoeffizienten der beiden homogenen Würfel ergibt sich unter Verwendung der
Werte für den hohlen Aluminiummantel
\begin{align*}
  \bar{\mu_2}&= \SI{0.614(42)}{1\per \centi\metre}\,, \\
  \bar{\mu_3}&=\SI{0.133(9)}{1\per \centi\metre} \,.
\end{align*}
Damit wird das Material in Würfel 2 zu Messing mit einer Abweichung von $1{,}57\%$
und das in Würfel 3 zu CH2O mit einer Abweichung von $14{,}75\%$ bestimmt. Es liegt
jedoch keine Information darüber vor, aus welchen Materialien die Würfel tatsächlich bestehen.
Es ist zudem anzumerken, dass der statistische Fehler in diesem Rechenschritt recht groß
wird, da die Nullmessungen für den hohlen Aluminiummantel nur 120\,s lang waren.

Die bestimmten Materialien des zusammengesetzten Würfels sind in Tabelle \ref{tab:ergebnisse2}
aufgeführt.

(UND DANN HIER NOCH WAS MAN DA SO ERKENNEN KANN...)
