\section{Diskussion}
\label{sec:Diskussion}

Zusammenfassend lässt sich der Versuch nur bedingt als erfolgreich bewerten.

Im Spektrum ist ein eindeutiger Peak zu sehen. Außerdem ist ein Compton-Spektrum
zu sehen, das aus der Wechselwirkung der Gammaphotonen mit dem Szintillator entsteht.
Somit kann das Spektrum schlüssig erklärt werden.

Die gemessenen Zählraten für die Nullmessung mit einem hohlen Aluminiummantel
ergeben sich zu
\begin{align*}
  I_{0,1}&=\SI{171.83(135)}{\per\second} \,, \\
  I_{0,2}&=\SI{169.63(133)}{\per\second} \,, \\
  I_{0,3}&=\SI{158.78(133)}{\per\second} \,.
\end{align*}

Für die Absorptionskoeffizienten der beiden homogenen Würfel ergibt sich unter Verwendung der
Werte für den hohlen Aluminiummantel
\begin{align*}
  \bar{\mu}_2&= \SI{0.614(42)}{1\per \centi\metre}\,, \\
  \bar{\mu}_3&=\SI{0.133(9)}{1\per \centi\metre} \,.
\end{align*}
Damit wird das Material in Würfel 2 zu Messing mit einer Abweichung von $1{,}57\%$
und das in Würfel 3 zu CH2O mit einer Abweichung von $14{,}75\%$ bestimmt. Es liegt
jedoch keine Information darüber vor, aus welchen Materialien die Würfel tatsächlich bestehen.
Es ist zudem anzumerken, dass der statistische Fehler in diesem Rechenschritt recht groß
wird, da die Nullmessungen für den hohlen Aluminiummantel nur 120\,s lang waren.

Die bestimmten Materialien des zusammengesetzten Würfels sind in Tabelle \ref{tab:ergebnisse2}
aufgeführt. Es ist erkennbar, dass die bestimmten Werte von den Theoriewerten teilweise sehr stark abweichen. Außerdem kann keine Aussage darüber getroffen werden, ob der Würfel tatsächlich aus den angenommenen Materialien zusammengesetzt ist. Ein möglicher Grund für die hohen Abweichungen ist, dass der
Strahl nicht perfekt punktförmig ist und damit die Messungen diagonal durch den Würfel ungenau sind, da der Strahl nicht nur von den theoretisch zu erwartenden Materialien abgeschwächt wird, sondern teilweise auch von den Materialien angrenzender Elementarwürfel. Zudem ist insbesondere bei den diagonalen Messungen auch eine ungenaue Justierung des Würfels ein möglicher Grund für Fehler.
