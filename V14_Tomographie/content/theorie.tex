\section{Theorie}
\label{sec:Theorie}

Die Tomographie nutzt als bildgebendes Verfahren die unterschiedliche Abschwächung von Strahlung in verschiedenen Materialien, aus denen ein Körper zusammengesetzt ist, aus, um Informationen über die Struktur des Körpers zu gewinnen.

\subsection{Entstehung und Mechanismen der Absorption von Gammastrahlung}
\label{subsec:theorie1}

In diesem Versuch wird dazu Gammastrahlung verwendet. Gammastrahlung entsteht beim radioaktiven Alpha- oder Betazerfall, wenn der Tochterkern aus seinem angeregten Zustand unter Abstrahlung eines Photons relaxiert. Diese haben eine diskrete Energie mit einer geringen Breite.
Als Beispiel kann das Nuklid $\ce{^{137}Cs}$ betrachtet werden, das durch Betazerfall zu $\ce{^{137}Ba}$ zerfällt, wobei ein Gammaphoton mit einer Energie von $\SI{0.6617}{\mega\electronvolt}$ abgestrahlt wird.

Prinzipiell sind drei verschiedene Mechanismen der Abschwächung von Gammastrahlung in Materie relevant: Bei niedrigen Energien und hohen Ordnungszahlen des Stoffes dominiert der Photoeffekt, bei dem ein Photon von einem Hüllenelektron des Absorbermaterials
absorbiert wird, wodurch dieses seine Energie erhält. Ist die Energie des Quants dabei höher als die Bindungsenergie des Elektrons, so kann es aus dem Atom herausgelöst werden. Die restliche Energie wird in kinetische Energie des Elektrons umgewandelt.

Beim Compton-Effekt wird ein Photon inelastisch an einem Elektron gestreut und gibt dort einen Teil seiner Energie ab, sodass das Elektron einen Impulsübertrag erfährt.
Verschiedene Photonen werden dabei in unterschiedliche Richtungen gestreut, was zu einer Intensitätsabnahme
des Strahls führt.
Der Energieübertrag $\Delta E = E - E'$ in Abhängigkeit des Streuwinkels des Photons $\theta$ kann mithilfe der Formel
\begin{equation}
	\frac{1}{E'} - \frac{1}{E} = \frac{1}{m_\text{e} c^2} (1-\cos\theta)
\end{equation}
bestimmt werden, wobei $m_\text{e}$ die Elektronenmasse und $c$ die Lichtgeschwindigkeit ist. Es ist ersichtlich, dass der Energieübertrag gegen 0 geht, wenn der Streuwinkel gegen 0 geht und maximal wird, wenn $\theta$ gegen $\SI{180}{\degree}$ geht. Somit gibt es eine Grenze an den Energieübertrag.
Der Compton-Effekt ist bei mittleren Energien der Photonen und kleinen Ordnungszahlen $Z$ des abschwächenden Materials dominant.

Bei hohen Quantenenergien tritt insbesondere die Paarbildung auf. Dabei können aus
einem Photon ein Elektron und ein Positron gebildet werden, wobei das
Photon verschwindet. Die Energie des Photons muss dafür mindestens das doppelte
der Ruheenergie eines Elektrons betragen. Dies ist der Grund dafür, dass der Effekt der Paarbildung in diesem Versuch nicht beachtet werden muss, da die Energie des Gammazerfalls von $\ce{^{137}Cs}$ mit $\SI{0.6617}{\mega\electronvolt}$ unter $2m_\text{e} \approx \SI{1.022}{\mega\electronvolt}$ liegt.

%In Abbildung \ref{fig:gamma} ist beispielhaft der Absorptionskoeffizient von Germanium in
%Abhängigkeit von der Energie der Strahlung dargestellt.
%
%\begin{figure}
%  \centering
%  \includegraphics[width=9cm]{data/germanium.png}
%  \caption{Absorptionskoeffizient von Germanium in Abhängikeit von der Energie
%  der Strahlung \cite{Versuchsanleitung}.}
%  \label{fig:gamma}
%\end{figure}

\subsection{Verfahren zur experimentellen Bestimmung der Abschwächungskoeffizienten}

Für die Absorption von Gammastrahlung in Materie gilt das Lambert-Beersche-Gesetz:
\begin{equation}
	N = I_0 \exp\left(-\sum_i \mu_i d_i\right)\,.
	\label{eqn:lambertBeer}
\end{equation}
Dabei wird die Zählrate $N$ nach Durchlauf verschiedener Materialien mit Dicken $d_i$ und Abschwächungskoeffizienten $\mu_i$ in Verbindung gesetzt, wobei $I_0$ die Zählrate ohne Absorber im Strahlengang bzw. vor dem Absorber ist.
Diese Gleichung lässt sich logarithmieren, um die Matrixgleichung
\begin{equation}
	\symbf{A} \cdot \vec{\mu} = \vec{I}
\end{equation}
zu erhalten, wobei die Matrix $\symbf{A}$ die Würfelgeometrie parametrisiert. $\vec{\mu} = (\mu_1, \dots, \mu_n)$ ist ein Spaltenvektor, der die Abschwächungskoeffizienten der durchlaufenden Schichten enthält und $\vec{I}$ ist ein Spaltenvektor mit $I_i = \ln(I_0/N_i)$. Die einzelnen $I_i$ werden in diesem Zusammenhang auch Projektionen genannt und beschreiben verschiedene Strahlengänge durch das zu untersuchende Material. In diesem Versuch wird zum Beispiel eine Ebene eines Würfels untersucht, der aus 27 Elementarwürfeln besteht, sodass $\vec{\mu}$ neun Einträge hat.

Die Matrix $\symbf{A}$ muss nicht notwendigerweise quadratisch sein. Es ist möglich, ein überbestimmtes Gleichungssystem zu erzeugen und dieses zu lösen, um die Messgenauigkeit zu erhöhen. Unter der Bedingung, dass $\symbf{A}^T \symbf{A}$ invertierbar ist, folgt für die aus den Messwerten für die Projektionen geschätzten Abschwächungskoeffizienten
\begin{equation}
	\hat{\vec{\mu}} = (\symbf{A}^T \symbf{A})^{-1} \symbf{A}^T \, \vec{I}\,,
	\label{eqn:mumatrix}
\end{equation}
was einer kleinste-Quadrate-Ausgleichsrechnung entspricht.
Die Kovarianzmatrix, die auf der Hauptdiagonalen die Varianzen der geschätzten Abschwächungskoeffizienten enthält, lässt sich durch
\begin{equation}
	V[\hat{\vec{\mu}}] = (\symbf{A}^T \symbf{A})^{-1} \symbf{A}^T \, V[\vec{I}] \, \symbf{A} (\symbf{A}^T \symbf{A})^{-1}
\end{equation}
aus der Kovarianzmatrix der Projektionen $V[\vec{I}]$ bestimmen.
