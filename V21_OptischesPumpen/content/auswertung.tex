\section{Auswertung}
\label{sec:Auswertung}

\subsection{Bestimmung des vertikalen Erdmagnetfeldes}
\label{subsec:vertikal}

Es wurde im Versuch mithilfe eines Helmholtzspulenpaares ein vertikales Magnetfeld
erzeugt, das die vertikale Komponente des Erdmagnetfeldes kompensierte. Die vertikale
Komponente des Erdmagnetfeldes lässt sich daher aus den Daten des Helmholtzspulenpaares
mithilfe der Formel
\begin{equation}
    B= \mu_0 \cdot  \frac{8}{\sqrt {125}}\cdot \frac{I\cdot N}{R}
    \label{eqn:Helmholtz}
\end{equation}
berechnen. Mit den Daten aus der Versuchsanleitung (ZITAT) und einem Strom von
$I_vert=0.195\,$A ergibt sich die vertikale Komponente des Erdmagnetfeldes zu
\begin{equation*}
  \SI{2.99e-5}{\tesla}\,.
\end{equation*}
Die Theoriewert hierzu ist (WERT + ZITAT).

\subsection{Bestimmung des horizontalen Erdmagnetfeldes und der Landé Faktoren}
\label{subsec:horizontal}

Zunächst werden mithilfe von Formel \eqref{eqn:Helmholtz} aus den gemessenen Stromstärken
die horizontalen Magnetfeldstärken berechnet. Die gemessenen und die daraus berechneten Werte befinden sich
in Tabelle (REFERENZ)

TABELLE EINFÜGEN!

Anschließend wird für beide Isotope getrennt jeweils die horizontale Magnetfeldstärke $B_hor$
gegen die Frequenz $f$ aufgetragen und es wird eine lineare Ausgleichsrechnung der Form
\begin{equation*}
  g(f)=af+b
\end{equation*}
durchgeführt. Es ergeben sich die Parameter
\begin{align*}
 irgendwas...
\end{align*}


\subsection{Bestimmung des Kernspins}
\label{subsec:Kernspin}

Für die beiden Rb Isotope gilt $J=\frac{1}{2}$ und $S=\frac{1}{2}$. Damit lässt sich
der Landé Faktor $g_J$ mithilfe von Formel (REFERNZ) bestimmen zu $g_J=2{,}0023$.
