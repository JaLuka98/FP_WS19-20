\newpage
\section{Diskussion}
\label{sec:Diskussion}

Der Versuch kann in Summe nur bedingt als erfolgreich betrachtet werden. Die Messwerte
liefern zwar Ergebnisse in der richtigen Größenordnung, allerdings weichen alle Ergebnisse um
etwa $25-35\%$ vom Theoriewert ab. Das weist auf grobe systematische Fehler bei der Messung hin.

Der berechnete Wert für die vertikale Komponente des Erdmagnetfeldes weicht mit
$b_{vert}=\SI{2.99e-5}{\tesla}$ um etwa $-32{,}05\%$ vom Literaturwert ab. Diese Abweichung
ist nur durch systematische Fehler oder grobe Messfehler zu erklären.

Ähnlich große Abweichungen zeigen sich für die Werte für die horizontale Komponente.
Die Werte betragen hier $B_{hor,1}=\SI{1.509(016)e-05}{\tesla}$ und
$B_{hor,1}=\SI{1.53(07)e-05}{\tesla}$ mit Abweichungen von $-24{,}55\%$ und $-23{,}5\%$
vom Theoriewert. Die Landé-Faktorn $g_{F,i}$ wurden zu
\begin{align*}
  g_{F,1}&= \SI{0.7041(0018)}{}\,,\\
  g_{F,2}&= \SI{0.473(004)}{}
\end{align*}
bestimmt. Sie weichen um $\%$ und $\%$ von den Theoriewerten ab.
Die daraus bestimmten Kernspins für die beiden Rubidiumisotope betragen
\begin{align*}
  I_1&= \SI{0.922(004)}{}\,,\\
  I_2&= \SI{1.616(017)}{}
\end{align*}
und haben eine relative Abweichung von $-35{,}36\%$ und $-38{,}53\%$ von den jeweiligen
Theoriewerten. Auffällig ist hier, dass die Unsicherheit der Werte extrem gering ist,
die Nominalwerte jedoch trotzdem eine sehr große Abweichung vom Theoriewert aufweisen.

Das Isotopenverhältnis im Gas konnte zu
\begin{equation}
  \frac{\gamma_{85}}{\gamma_{87}}= 2
\end{equation}
bestimmt werden. Die Diskrepanz zum in der Natur vorkommenden Verhältnis von
etwa 2{,}593 ist dadurch zu erklären, dass GRÜNDE EINTRAGEN!!!

Eine Abschätzung des Beitrages des quadratischen Zeeman-Effekts konnte zeigen, dass
dieser im gegebenen Versuch nur eine stark untergeordnete Rolle spielt. Der Beitrag des
quadratischen Terms liegt um etwa fünf Größenordnungen unter dem des linearen Terms.

Es konnte gezeigt werden, dass die Zunahme der Transparenz sich gut durch eine Exponentialfunktion
annähern lässt. Außerdem konnte aus den Rabi-Oszillationen das Verhältnis der Landé-Faktoren zueinander
zu
\begin{equation*}
  \frac{b_2}{b_1}\approx\SI{1.13(17)}{}
\end{equation*}
bestimmt werden. Die Abweichung vom Theoriewert 1{,}5 beträgt hier $-24{,}67\%$.
Auch hier liegt der Theoriewert nicht in der Umgebung einer Standardabweichung um den Wert, sondern
deutlich davon entfernt.

Zusammenfassend sind alle Ergebnisse zwar in der richtigen Größenordnung, weisen jedoch alle
Abweichungen von etwa $25-35\%$ vom Theoriewert auf. Zudem sind die Unsicherheiten der berechneten
Werte sehr gering, sodass die Abweichungen von den Theoriewerten nicht durch statistische
Schwankungen erklärt werden können. Daher liegt der Schluss nahe, dass hier systematische Fehler
oder grobe Messfehler vorliegen.
