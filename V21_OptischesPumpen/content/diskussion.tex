\newpage
\section{Diskussion}
\label{sec:Diskussion}

Der Versuch kann in Summe nur bedingt als erfolgreich betrachtet werden. Die Messwerte
liefern zwar Ergebnisse in der richtigen Größenordnung, allerdings weichen alle Ergebnisse um
etwa $25-35\%$ vom Theoriewert ab. Das weist auf grobe systematische Fehler bei der Messung hin.

Der berechnete Wert für die vertikale Komponente des Erdmagnetfeldes weicht mit
$B_{\text{vert}}=\SI{2.99e-5}{\tesla}$ um etwa $-32{,}05\%$ vom Literaturwert ab. Diese Abweichung
ist nur durch systematische Fehler oder grobe Messfehler zu erklären.

Ähnlich große Abweichungen zeigen sich für die Werte für die horizontale Komponente.
Die Werte betragen hier $B_{\text{hor},1}=\SI{1.509(016)e-05}{\tesla}$ und
$B_{\text{hor},1}=\SI{1.53(07)e-05}{\tesla}$ mit Abweichungen von $-24{,}55\%$ und $-23{,}5\%$
vom Theoriewert. Die berechneten Landé-Faktoren $g_{F,i}$ sind
\begin{align*}
  g_{F,1}&= \SI{0.7041(0018)}{}\,,\\
  g_{F,2}&= \SI{0.473(004)}{}\,.
\end{align*}
Die daraus bestimmten Kernspins für die beiden Rubidiumisotope betragen
\begin{align*}
  I_1&= \SI{0.922(004)}{}\,,\\
  I_2&= \SI{1.616(017)}{}
\end{align*}
und haben eine relative Abweichung von $-35{,}36\%$ und $-38{,}53\%$ von den jeweiligen
Theoriewerten. Auffällig ist hier, dass die Unsicherheit der Werte extrem gering ist,
die Nominalwerte jedoch trotzdem eine sehr große Abweichung vom Theoriewert aufweisen.
Außerdem fällt auf, dass die Messwerte, die zur Bestimmung von den $g_{F,i}$ verwendet werden
sehr gut auf den Ausgleichsgeraden liegen. Trotzdem sind die aus den Geraden berechneten Werte
für die $g_{F,i}$, wie an den daraus berechneten Kernspins zu sehen ist, deutlich zu niedrig.
Das bedeutet, dass die Steigung der Geraden zu gering ausfällt. Physikalisch bedeutet das, dass
die aus den gemessenen Strömen berechneten Magnetfelder hier systematisch unterschätzt werden.

Das bestimmte Isotopenverhältnis im Gas beträgt
\begin{equation}
  \frac{\gamma_{85}}{\gamma_{87}}= 2 \,.
\end{equation}
Die Diskrepanz zum in der Natur vorkommenden Verhältnis von
etwa 2{,}593 beträgt $-22{,}87\%$.

Eine Abschätzung des Beitrages des quadratischen Zeeman-Effekts zeigt, dass
dieser im gegebenen Versuch nur eine stark untergeordnete Rolle spielt. Der Beitrag des
quadratischen Terms liegt um etwa fünf Größenordnungen unter dem des linearen Terms.

Die Zunahme der Transparenz lässt sich gut durch eine Exponentialfunktion
annähern. Die Messwerte folgen hier sehr gut dem Verlauf der Ausgleichsfunktion.
Der aus den Rabi-Oszillationen berechnete Wert für das Verhältnis der Landé-Faktoren zueinander
beträgt
\begin{equation*}
  \frac{b_2}{b_1}\approx\SI{1.13(17)}{} \,.
\end{equation*}
Die Abweichung vom Theoriewert 1{,}5 beträgt hier $-24{,}67\%$.
Auch hier liegt der Theoriewert nicht in der Umgebung einer Standardabweichung um den Wert, sondern
deutlich davon entfernt.

Zusammenfassend sind alle Ergebnisse zwar in der richtigen Größenordnung, weisen jedoch alle
Abweichungen von etwa $25-35\%$ vom Theoriewert auf, die aus einer Unterschätzung des B-Feldes resultieren.
Zudem sind die Unsicherheiten der berechneten
Werte sehr gering, sodass die Abweichungen von den Theoriewerten nicht durch statistische
Schwankungen erklärt werden können. Daher liegt der Schluss nahe, dass hier systematische Fehler
oder grobe Messfehler vorliegen.
