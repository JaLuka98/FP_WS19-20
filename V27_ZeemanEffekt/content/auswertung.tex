\section{Auswertung}
\label{sec:Auswertung}

\subsection{Interpolation zur Parametrisierung des Magnetfeldes}

Zur Auswertung des Versuchs und Durchführung der Rechnungen muss das Magnetfeld $B$
an der Stelle der Probe bekannt sein. Allerdings ist die Messung mit der vorliegenden Hall-Sonde
ungenau, weil die Stelle der Messung nicht immer genau gleich eingestellt werden kann.
Außerdem liegen natürliche Fluktuationen des Magnetfelds durch andere Experimente und elektromagnetische
Störungen durch Umgebungseinflüsse vor. Deswegen wird eine lineare Ausgleichsrechnung der Form

\begin{equation*}
  I(B) = aI+b
\end{equation*}

angesetzt, um einen eingestellten Strom $I$ in ein Magnetfeld $B$ mit einer Abschätzung der Unsicherheit umzurechnen.
Die dafür verwendeten Werte sind in Tabelle \ref{tab:IB} zu finden, wobei die angezeigten Nachkommastellen der magnetischen Flussdichte beim Messgerät nicht berücksichtigt wurden, da sie stark fluktuierten.

\begin{table}[htp]
	\begin{center}
    \caption{Eingestellte Stromstärken $I$ und gemessene magnetische Flussdichten $B$ zur Interpolation für die Parametrisierung der Flussdichte}
    \label{tab:IB}
		\begin{tabular}{cc}
		\toprule
			{$I$/A} & {$B$/mT}\\
			\midrule
			 0.0 &    5\\
			 0.5 &   48\\
			 1.0 &   72\\
			 1.5 &  104\\
			 2.0 &  134\\
			 2.5 &  175\\
			 3.0 &  196\\
			 3.5 &  236\\
			 4.0 &  228\\
			 4.5 &  295\\
			 5.0 &  308\\
			 5.5 &  360\\
			 6.0 &  393\\
			 6.5 &  410\\
			 7.0 &  460\\
			 7.5 &  474\\
			 8.0 &  499\\
			 8.5 &  514\\
			 9.0 &  568\\
			 9.5 &  572\\
			10.0 &  620\\
			10.5 &  637\\
			11.0 &  659\\
			11.5 &  692\\
			12.0 &  722\\
			12.5 &  776\\
			13.0 &  824\\
			13.5 &  909\\
			14.0 &  940\\
			14.5 &  920\\
			15.0 &  968\\
			15.5 & 1040\\
			16.0 & 1073\\
			16.5 & 1088\\
			17.0 & 1099\\
			17.5 & 1127\\
			18.0 & 1126\\
		\bottomrule
		\end{tabular}
	\end{center}
\end{table}

Die Ausgleichsrechnung ergibt konkret die Parameter

\begin{align}
  a &= \SI{0.0643(8)}{\tesla\per\ampere}\,,\\
  b &= \SI{-0.004(8)}{\tesla}\,.
\end{align}

Die Ausgleichsfunktion mit diesen konkreten Parametern wird in der Folge verwendet, um
die am Ort der Probe vorhandenen magnetischen Feldstärken in Abhängigkeit der eingestellten Stromstärken zu bestimmen.

\subsection{Bestimmung des Landé-Faktors für die rote Linie}

Für die rote Linie wurden vier Bilder aufgenommen: Zwei mal mit angelegtem Magnetfeld und zwei mal ohne, jeweils für die Einstellungen des Polarisationsfilters \ang{0} und \ang{90}.

\begin{table}[htp]
	\begin{center}
    \caption{\cite{insert caption}}
    \label{tab:rot}
		\begin{tabular}{ccc}
		\toprule
			{$\Delta s$/px} & {$\delta s$/px} & {$\delta \lambda$/pm}\\
			\midrule
			137 &  56 & 10.00\\
			152 &  69 & 11.10\\
			157 &  70 & 10.90\\
			170 &  67 &  9.64\\
			171 &  72 & 10.30\\
			180 &  78 & 10.60\\
			192 &  83 & 10.57\\
			206 &  88 & 10.45\\
			221 &  90 &  9.96\\
			238 & 101 & 10.38\\
		\bottomrule
		\end{tabular}
	\end{center}
\end{table}
