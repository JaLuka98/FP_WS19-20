\section{Diskussion}
\label{sec:Diskussion}

Die Parametrisierung der magnetischen Flussdichte als Funktion der Stromstärke ist erfolgreich, darauf weisen einerseits die geringen Fehler der Parameter \eqref{eqn:paramsB(I)} und andererseits der gute grafische Ausgleich in Abbildung \ref{fig:bAusgleich} hin.

Zur Beurteilung der Messung der Landé-Faktoren sind diese in Tabelle \ref{tab:zusammenfassung} zusammengefasst.

\begin{table}[htp]
	\begin{center}
    \caption{Zusammenfassung der gemessenen Landé-Faktoren, den theoretischen Vorhersagen und den relativen Abweichungen.}
    \label{tab:zusammenfassung}
		\begin{tabular}{cccc}
		\toprule
			{Farbe/Polarisation} & {$g_{J,\text{exp}}$} & {$g_{J,\text{theo}}$} & {Rel. Abw.}\\
			\midrule
			 rot/$\sigma$  & $\num{0.964(23)}$ & 1 & $-3{,}6\%$\\
			 blau/$\pi$    & $\num{0.613(16)}$ & 0,5 & $22{,}6\%$\\
       blau/$\sigma$ &  $\num{1.50(5)}$  & 1,75 & $-14{,}29\%$\\
		\bottomrule
		\end{tabular}
	\end{center}
\end{table}

Es ist zu erkennbar, dass die Messung des Faktors der roten Linie am genauesten ist, während die Theoriewerte bei den Messungen der Faktoren für die blaue Linie zwar außerhalb einiger Standardfehlerumgebungen der Messwerte liegen. Dennoch sind die Messungen grundsätzlich als erfolgreich einzustufen, da sich die erwarteten Linienaufspaltungen in den Bildern ergeben und die Messungen qualitativ im richtigen Bereich liegen. Die Vermessung der blauen $\SI{480}{\nano\meter}$ wird zusätzlich dadurch erschwert, dass Cadmium über zwei intensive blaue Linien verfügt, die im Versuch nicht eindeutig scharf zu trennen waren. Zusätzlich mit der fehlenden Trennschärfe der beiden $\sigma$-Übergänge mit anderen Übergangsenergien und dem manuellen Ablesen erklärt dies die vorhandenen Abweichungen.
