\newpage
\section{Diskussion}
\label{sec:Diskussion}

Insgesamt ist der Versuch nur bedingt als erfolgreich zu bewerten.

Die berechnete Foliendicke $\Delta x = \SI{4.4(4)}{\micro\metre}$ weicht um
$120\%$ vom wahren Wert $\Delta x = \SI{2}{\micro\metre}$ ab. Ein möglicher Grund für diese
Abweichung ist, dass die Ausgleichsgeraden für die Messreihen mit und ohne Folie
nicht parallel sind. Dadurch sind auch die für die Berechnung der Foliendicke verwendeten
Y-Achsenabschnitte deutlich weiter auseinander als wenn die Geraden parallel wären,
wie es auch theoretisch zu erwarten wäre. Außerdem ist anzumerken, dass
bei der Messreihe ohne Folie der Wertebereich, für den die Ausgleichsrechnung durchgeführt wird,
manuell per Augenmaß bestimmt wurde.

Die heutige Aktivität der Probe kann zu $A_{\text{heute,theo}}= \SI{316.94}{\kilo\becquerel}$
bestimmt werden. Der Experimentell bestimmte Wert
$A_{\text{heute,exp}}=\SI{83.80(173)}{\kilo\becquerel}$ weicht davon um
$-73{,}56\%$ ab. Ein möglicher Grund für diese Abweichung ist, dass der Kammerdruck
während der gesamten Messung mit etwa $110-120$\,mbar deutlich über dem theoretisch
angenommenen Wert von $0$\,mbar liegt. Durch den nichtverschwindenden Kammerdruck
könnten somit zusätzliche Streuungen der $\alpha$-Teilchen und ein zusätzlicher
Energieverlust herbeigeführt werden. Ein weiterer möglicher Grund ist, dass der Detektor
schlecht positioniert sein könnte, wodurch die berechnete effektive Detektorfläche
nicht der realen entspricht.

Die aus den Messwerten berechneten Wirkungsquerschnitte folgen, wie in Abbildung
\ref{fig:wq} zu sehen ist, nicht dem theoretisch zu erwartenden Verlauf. Mögliche Gründe
dafür sind, dass die rutherford'sche Streuformel selbst nur eine Näherung ist und
dass die Zählraten bei der Messung durch den nichtverschwindenden Kammerdruck
verfälscht werden.

Bei der Berechnung der Wirkungsquerschnitte für zwei Folien verschiedener Dicken
ergeben sich die Werte
\begin{align*}
  \left(\frac{d \sigma}{d \Omega}\right)_2&=\SI{3.92(11)e-21}{\metre\squared} \,, \\
  \left(\frac{d \sigma}{d \Omega}\right)_4&=\SI{1.79(05)e-21}{\metre\squared} \,.
\end{align*}

Für die Abhängigkeit des Wirkungsquerschintts von der Kernladungszahl konnte experimentell
nur gezeigt werden, dass mit höheren Kernladungszahlen auch höhere Wirkungsquerschnitte
einhergehen. Der theoretisch erwartete quadratische Verlauf konnte aufgrund von zu
wenigen Messdaten nicht nachvollzogen werden. Außerdem liegen die experimentell
bestimmten Werte etwa zwei Größenordnungen über den theoretisch bestimmten.


%Kammerdruck nicht bei 0 \\
%  -> die Energien sind viel geringer als theoretisch angenommen \\
%  -> die Aktivität ist nicht so hoch wie sie eigentlich sein sollte
%
%bei den Messungen mit z abhängigkeit und mehrfachstreuung kann eine exakt gleiche montage nicht garantiert werden.
%
%
%Bereiche für die die Geraden gefittet werden werden per Augenmaß bestimmt.
%
%Geraden nicht parallel -> Y-Achsenabschnitte weiter auseinander als wenn sie parallel wären
%
