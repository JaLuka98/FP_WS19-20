\section{Auswertung}
\label{sec:Auswertung}


(HIER DIE AUSWERTUNG DES PEAKS)

\subsection{Bestimmung der Foliendicke}
\label{subsec:dicke}

Die zu Bestimmung der Foliendicke verwendeten Messdaten für die Pulshöhe
mit und ohne Folie befinden sich in Tabelle \ref{tab:dicke}.

\begin{table}[htp]
	\begin{center}
    \caption{Messwerte zur Bestimmung der Foliendicke.}
    \label{tab:dicke}
		\begin{tabular}{cccc}
		\toprule
			{$p_\text{Folie}$/mbar} & {$U_{\text{Folie}}$/V} & {$p$/mbar} & {$U$/V}\\
			\midrule
			107 & 9,84 & 118 & 11,8\\
			114 & 9,52 & 113 & 11,9\\
			126 & 8,96 & 130 & 11,4\\
			140 & 8,24 & 140 & 11,3\\
			148 & 7,20 & 150 & 11,0\\
			156 & 6,88 & 160 & 10,9\\
			163 & 6,32 & 170 & 10,5\\
			177 & 5,84 & 180 & 10,4\\
			187 & 5,20 & 190 & 10,5\\
			199 & 4,08 & 200 & 10,0\\
			- & -  & 213 & 7,9\\
			- & -  & 225 & 8,0\\
			- & -  & 240 & 7,1\\
			- & -  & 250 & 6,0\\
			- & -  & 260 & 4,5\\
			- & -  & 275 & 3,5\\
			- & -  & 285 & 2,6\\
			- & -  & 300 & 2,5\\
		\bottomrule
		\end{tabular}
	\end{center}
\end{table}

Es wird die Pulshöhe in Abhängigkeit vom Druck aufgetragen und für den
linear abfallenden Bereich eine Ausgleichsrechnung der Form
\begin{equation}
  f(p)=ap+b
\end{equation}
durchgeführt. Dies ist in Abbildung \ref{fig:dicke} zu sehen. Bei der linearen Ausgleichsrechnung
für die Messreihe ohne Folie werden dabei nur die in blau dargestellten Messwerte berücksichtigt.

\begin{figure}
  \centering
  \includegraphics[width=\textwidth]{build/dicke.pdf}
  \caption{Messwerte für die Messreihen mit und ohne Folie, sowie lineare Ausgleichsrechnungen.}
  \label{fig:dicke}
\end{figure}

Es ergeben sich die Parameter
\begin{align}
  a_{\text{Folie}}&=\SI{}{\volt\per\milli\bar} \, \\
  b_{\text{Folie}}&=\SI{}{\volt} \,\\
  a&=\SI{}{\volt\per\milli\bar} \, \\
  b&=\SI{}{\volt} \,.
\end{align}

(HIER DANN NOCH DAS MIT BREMSVERMÖGEN UND DER GANZE SPAß)

\subsection{Bestimmung des Raumwinkels}
\label{subsec:raumwinkel}

Der Raumwinkel $\Delta \Omega$ ist der Raumwinkel um die Folie herum, der die
effektive Detektorfläche beschreibt. Die effektive Detektorfläche folgt aus geometrischen
Überlegungen. Diese sind in Abbildung (REFERENZ) veranschaulicht.

(BILD MIT SKZIZZEN EINFÜGEN!)

Aus trigonometrischen Verhältnissen kann hergeleitet werden, dass gilt:
\begin{align*}
  b &= \frac{b_\symup{b} \cdot d}{c} \,,\\
  h &= \frac{h_\symup{b} \cdot d}{c} \,.
\end{align*}
Dabei bezeichnet $b_\symup{b}$ die Breite und $h_\symup{b}$ die Höhe der Blende.
Die Breite der effektiven Detektorfläche wird mit $b$ bezeichnet und die entsprechende Höhe mit $h$.
Der Abstand von der Folie zur Blende ist durch $c$ gegeben und $d$ beschreibt den Abstand
der Folie und der Detektoroberfläche. Aus den bekannten Werten (ZITAT VERSUCHSANLEITUNG)
\begin{align*}
  b_\symup{b} = \SI{2}{\milli\meter} \,, \\
  h_\symup{b} = \SI{10}{\milli\meter} \,, \\
  c = \SI{41}{\milli\meter} \,,\\
  d = \SI{41}{\milli\meter}
\end{align*}
lassen sich die Werte
\begin{align*}
  b &= \SI{2.195}{\milli\meter} \,,\\
  h &= \SI{10.976}{\milli\meter}
\end{align*}
berechnen. Diese charakterisieren die effektive Detektorfläche. Der Raumwinkel kann dann durch
\begin{equation}
  \Delta \Omega = 4 \arctan\left(\frac{b h}{2d \sqrt{4d^2 + b^2 + h^2}}\right)
\end{equation}
berechnet werden. Mit den oben genannten Werten ergibt sich
\begin{equation*}
  \Delta \Omega = \SI{}{sr} \,.
\end{equation*}

Der Raumwinkel um die Quelle herum folgt mit den Werten (ZITAT VERSUCHANLEITUNG)
\begin{align*}
  c &= \SI{97}{\milli\meter} \,, \\
  d &= \SI{101}{\milli\meter}
\end{align*}
zu
\begin{equation*}
  \Delta \Omega_\text{Quelle} = \SI{}{sr} \,.
\end{equation*}

\subsection{Bestimmung der Aktivität der Probe}
\label{subsec:aktivitaet}

Die theoretisch heute zu erwartende Aktivität lässt sich mit Kenntnis der Halbwertszeit
und der Aktivität zu einem früheren Zeitpunkt berechnen. Im Oktober 1994 betrug die
Aktivität der Probe $\SI{330}{\kilo\becquerel}$ (ZITAT VERSUCHSANLEITUNG). Die Halbwertszeit
von 241-Americium beträgt $T_{1/2}=432$ Jahre. Daraus lässt sich mithilfe von
\begin{align*}
  A_{\text{heute,theo}}&= A_{1994}\exp(-\lambda*t) \, \\
  \lambda= \frac{\ln(2)}{T_{1/2}}
\end{align*}
die heutige Aktivität bestimmen. Die Zeit beträgt dabei $t=25+1/6$ Jahre. Somit beträgt die
theoretisch zuerwartende heutige Aktivität
\begin{equation*}
  A_{\text{heute,theo}}= \SI{}{\kilo\becquerel} \,.
\end{equation*}

Aus den Messwerten lässt sich die heutige Aktivität unter
Kenntnis des Raumwinkels $\Delta \Omega_{\text{Quelle}}$ aus der Messung der Zählrate berechnen.
Das Verhältnis der heutigen Aktivität und die gemessene Zählrate müssen nämlich in
dem gleichen Verhältnis zueinander stehen, wie der Raumwinkel $\Delta \Omega_{\text{Quelle}}$ und
der gesamte Raumwinkel. Somit ist die heutige Aktivität gegeben durch
\begin{equation*}
  A_{\text{heute,exp}}=\frac{4 \pi  I}{\Delta \Omega_{\text{Quelle}}}=\SI{}{\kilo/becquerel} \,.
\end{equation*}


\subsection{Bestimmung des differentiellen Wirkungsquerschnitts}
\label{subsec:wq}

In Tabelle (REFERENZ) sind die Messwerte für die Zählrate in Abhängigkeit vom
Winkel aufgeführt. Außerdem befinden sich dort die berechneten differentiellen
Wirkungsquerschnitte. Diese werden wie folgt berechnet:

(TABELLE EINFÜGEN)

Für den Wirkungsquerschnitt gilt:
\begin{equation}
  \frac{\symup{d}\sigma}{\symup{d}\Omega} = \frac{I}{I_0  N_\symup{Au} \Delta x \Delta \Omega} \,.
  \label{eqn:wq}
\end{equation}
Dabei ist $I$ die Zählrate mit Folie, $I_0$ die Zählrate ohne Folie, $ N_\symup{Au}$
die Teilchendichte von Gold, $\Delta x$ die Foliendicke und $\Delta \Omega$ der Raumwinkel.
Die Teilchendichte von Gold lässt sich durch
\begin{equation*}
  N_\symup{Au} = N_\symup{A} \frac{\rho_\symup{Au}}{M_\symup{Au}}
\end{equation*}
berechnen. Mit den Werte (WERTE EINFÜGEN UND ZITIEREN) ergibt sich damit der Wert
\begin{equation*}
  N_\symup{Au} =
\end{equation*}
für die Teilchendichte von Gold.
Mit der Zählrate $I_0=$ ohne Folie, der Schichtdicke $\Delta x =2$\,µm und dem Raumwinkel
aus Kapitel \ref{subsec:raumwinkel} ergeben sich damit die in Tabelle (REFERENZ) dargestellten Werte.

Diese sind in Abbildung \ref{fig:wq} gegen den Streuwinkel $\theta$ aufgetragen. Außerdem ist
eine Theoriekurve zu sehen. Diese wird gemäß Gleichung \eqref{eqn:rutherford} mit den
Werten $z=2$ und $Z=79$ berechnet.

\begin{figure}
  \centering
  \includegraphics[width=\textwidth]{build/wq.pdf}
  \caption{Berechnete Werte für den differentiellen Wirkungsquerschnitt und Theoriekurve.}
  \label{fig:wq}
\end{figure}


\subsection{Untersuchung von Mehrfachstreuung}
\label{subsec:mehrfach}

Für zwei Goldfolien mit den Dicken $\Delta x=2$\,µm und $\Delta x=4$\,µm werden
die Zählraten
\begin{align*}
  I_2=
  I_4=
\end{align*}
gemessen. Analog zu der Berechnung in Kapitel \ref{subsec:wq} lassen dich die Wirkungsquerschnitte
\begin{align*}
  Werte..
\end{align*}
berechnen.

\subsection{Untersuchung der $Z$-Abhängigkeit}
\label{subsec:z}

Zur Untersuchung der Mehrfachstreuung wurden Folien aus Gold, Platin und
Bishmuth untersucht. In Tabelle (REFERENZ) sind die Dicken $\Delta x$ der Folien,
die Ordnungszahlen $Z$, die Dichten $\rho$, die molaren Massen $M$ und die daraus berechneten
Parameter $\frac{I}{N \Delta x}$ aufgeführt. $N$ bezeichnet hier die in Kapitel \ref{subsec:wq}
bereits eingeführte Teilchendichte des jeweiligen Elements.

(TABELLE EINFÜGEN)

In Abbildung \ref{fig:z} ist der Parameter $\frac{I}{N \Delta x}$ gegen die Ordnungszahl
aufgetragen. Es sind dabei zwei Werte bei $Z=79$ zu sehen, da zwei unterschiedlich dicke
Goldfolien untersucht wurden. Es ist zu erkennen, dass der Parameter $\frac{I}{N \Delta x}$
tendenziell mit der Ordnungszahl zunimmt.

\begin{figure}
  \centering
  \includegraphics[width=\textwidth]{build/z.pdf}
  \caption{Auftragung des Parameters $\frac{I}{N \Delta x}$ gegen die Ordnungszahl $Z$.}
  \label{fig:z}
\end{figure}
