\section{Auswertung}
\label{sec:Auswertung}

Im Folgenden werden alle Berechnungen in Python 3.7.1, unterstützt durch das
Paket NumPy \cite{numpy}, durchgeführt. Für Ausgleichsrechnungen wird SciPy
\cite{scipy} verwendet. Die Abbildungen werden mit matplotlib \cite{matplotlib} erstellt.

\subsection{Bestimmung der Pulshöhe und Anstiegszeit}
\label{subsec:puls}

In Abbildung \ref{fig:puls} ist beispielhaft ein auf dem Oszilloskop beobachteter Puls
dargestellt. Aus der Abbildung können die Werte
\begin{align*}
  U_\text{max}&= \SI{11.8}{\volt}\,,\\
  T_\text{Anstieg}&= \SI{1}{\micro\second}
\end{align*}
mithilfe des Cursors und der Zeiteinteilung abgelesen werden.

\begin{figure}
  \centering
  \includegraphics[width=0.75\textwidth]{images/peak.JPG}
  \caption{Darstellung eines Pulses auf dem Oszilloskop.}
  \label{fig:puls}
\end{figure}

\subsection{Bestimmung der Foliendicke}
\label{subsec:dicke}

Die zu Bestimmung der Foliendicke verwendeten Messdaten für die Pulshöhe
mit und ohne Folie befinden sich in Tabelle \ref{tab:dicke}.

\begin{table}[htp]
	\begin{center}
    \caption{Messwerte zur Bestimmung der Foliendicke. Die Messwerte mit Folie sind indiziert, während die Messwerte ohne eingebaute Folie keinen Index haben.}
    \label{tab:dicke}
		\begin{tabular}{cccc}
		\toprule
			{$p_\text{Folie}$/mbar} & {$U_{\text{Folie}}$/V} & {$p$/mbar} & {$U$/V}\\
			\midrule
			107 & 9,84 & 118 & 11,8\\
			114 & 9,52 & 113 & 11,9\\
			126 & 8,96 & 130 & 11,4\\
			140 & 8,24 & 140 & 11,3\\
			148 & 7,20 & 150 & 11,0\\
			156 & 6,88 & 160 & 10,9\\
			163 & 6,32 & 170 & 10,5\\
			177 & 5,84 & 180 & 10,4\\
			187 & 5,20 & 190 & 10,5\\
			199 & 4,08 & 200 & 10,0\\
			- & -  & 213 & 7,9\\
			- & -  & 225 & 8,0\\
			- & -  & 240 & 7,1\\
			- & -  & 250 & 6,0\\
			- & -  & 260 & 4,5\\
			- & -  & 275 & 3,5\\
			- & -  & 285 & 2,6\\
			- & -  & 300 & 2,5\\
		\bottomrule
		\end{tabular}
	\end{center}
\end{table}

Es wird die Pulshöhe in Abhängigkeit vom Druck aufgetragen und für den
linear abfallenden Bereich eine Ausgleichsrechnung der Form
\begin{equation*}
  f(p)=ap+b
\end{equation*}
durchgeführt. Dies ist in Abbildung \ref{fig:dicke} zu sehen. Bei der linearen Ausgleichsrechnung
für die Messreihe ohne Folie werden dabei nur die in blau dargestellten Messwerte berücksichtigt.

\begin{figure}
  \centering
  \includegraphics[width=\textwidth]{build/dicke.pdf}
  \caption{Messwerte für die Messreihen mit und ohne Folie, sowie lineare Ausgleichsrechnungen.}
  \label{fig:dicke}
\end{figure}

Es ergeben sich die Parameter
\begin{align*}
  a_{\text{Folie}}&=\SI{-0.0620(002)}{\volt\per\milli\bar} \, \\
  b_{\text{Folie}}&=\SI{16.61(35)}{\volt} \,\\
  a&=\SI{0.078(005)}{\volt\per\milli\bar} \, \\
  b&=\SI{25.3(11)}{\volt} \,.
\end{align*}

Die Parameter $b$ bzw. $b_{\text{Folie}}$ sind dann die geschätzten Spannungen bei einem vollständig evakuiertem Versuchsaufbau ohne bzw. mit Folie im Strahlengang.
Das Durchlaufen der Folie bedeutet für die Alphateilchen eine Energiedifferenz $\Delta E = E-E_{\text{Folie}}$ bezüglich des Durchlaufens mit bzw. ohne Folie, wobei $E = E_\alpha$ bzw. $E_{\text{Folie}}$ die Energien der Alphateilchen ohne bzw. mit Folie im Weg zum Detektor bezeichnen.
Um nun von Spannungen auf Energien zu schließen, verwendet man den Quotient aus den beiden $y$-Achsenabschnitten, für den
\begin{equation*}
  \frac{b_{\text{Folie}}}{b} = \frac{E_{\text{Folie}}}{E_\alpha}
\end{equation*}
gilt. Mit dieser Information lässt sich die Energiedifferenz durch den Zusammenhang
\begin{equation*}
  \Delta E = E_\alpha \left( 1 - \frac{b_{\text{Folie}}}{b} \right)
\end{equation*}
berechnen. Für die vorliegenden Werte ergibt sich $\Delta E = \SI{1.88(18)}{\mega\electronvolt}$.
Die Unsicherheit der Energiedifferenz kann mit
\begin{equation*}
  \sigma_{\Delta E} = \frac{E_\alpha}{b^2} \sqrt{\left( b \sigma_{b_\text{Folie}} \right)^2 + \left( b_\text{Folie} \sigma_b \right)^2}
\end{equation*}
berechnet werden \footnote{Ist $f$ eine Funktion, die von unsicheren Variablen $x_i$ mit
Standardabweichungen $\sigma_i$ abhängt, so ist die Unsicherheit von f
\begin{equation}
  \sigma_f = \sqrt{
    \sum\limits_{i = 1}^N
      \left( \frac{\partial f}{\partial x_i} \sigma_i \right)^{\!\! 2}
  }\,.
  \label{eqn:gaussfehler}
\end{equation}
Diese Formel wird auch "Gauß'sches Fehlerfortpflanzungsgesetz" gennant und hier verwendet.}.

Die Foliendicke $\Delta s$ wird berechnet, indem die Bethe-Bloch-Formel \eqref{eqn:bethe} als diskreter Energieverlust $\Delta E/\Delta x$ genähert wird.
Dann ergibt sich die Foliendicke durch Umstellen der Formel zu
\begin{equation}
  \Delta x = \Delta E
  \frac{m_{\mathrm{e}} v^2(4 \pi \epsilon_{\mathrm{0}})^2}
  {4\pi e^4 z^2 n_\text{Au} \ln{\frac{2 m_{\mathrm{e}} v^2}{I}}}
  \label{eqn:foliendicke}
\end{equation}
Da in dieser Formel nur die Energiedifferenz fehlerbehaftet ist, ist die Unsicherheit der Foliendicke durch
\begin{equation*}
  \sigma_{\Delta x} = \sigma_{\Delta E}
  \frac{m_{\mathrm{e}} v^2(4 \pi \epsilon_{\mathrm{0}})^2}
  {4\pi e^4 z^2 n_\text{Au} \ln{\frac{2 m_{\mathrm{e}} v^2}{I}}}
\end{equation*}
gegeben.

Die Teilchendichte von Gold lässt sich durch
\begin{equation*}
  N_{\symup{Au}} = N_{\symup{A}} \frac{\rho_{\symup{Au}}}{M_{\symup{Au}}}
\end{equation*}
berechnen. Mit den Werten $\rho_{\symup{Au}}=\SI{19.32}{\gram\per\cubic\centi\metre}$ \cite{rho}
und ${M_{\symup{Au}}}=\SI{196.97}{\gram\per\mol}$ \cite{molmasse}
ergibt sich der Wert
\begin{equation*}
  N_{\symup{Au}} \approx \SI{5.91e28}{\per\cubic\metre}\,.
\end{equation*}
Die Elektronendichte beträgt dann
\begin{equation}
  n_\text{Au} = N_\text{Au} \cdot Z = N_\text{Au} \cdot 79 \approx \SI{4.66e30}{\per\cubic\metre}\,,
\end{equation}
da jedes Goldatom im Mittel 79 Elektronen enthält, an denen die Alphateilchen streuen können.

Insgesamt beträgt die experimentell bestimmte Foliendicke dann
\begin{equation*}
  \Delta x = \SI{4.4(4)}{\micro\metre}
\end{equation*}
mit einer relativen Unsicherheit von $9{,}09\%$.
Die Herstellerangabe der Foliendicke ist $\SI{2}{\micro\meter}$. Die relative Abweichung zu dieser beträgt $120\%$.

\subsection{Bestimmung des Raumwinkels}
\label{subsec:raumwinkel}

Der Raumwinkel $\Delta \Omega$ ist der Raumwinkel um die Folie herum, der die
effektive Detektorfläche beschreibt. Die effektive Detektorfläche folgt aus geometrischen
Überlegungen. Diese sind in Abbildung \ref{fig:skizze} veranschaulicht.
Aus trigonometrischen Verhältnissen kann hergeleitet werden, dass gilt:
\begin{align*}
  b &= \frac{b_{\symup{b}} d}{c} \,,\\
  h &= \frac{h_{\symup{b}} d}{c} \,.
\end{align*}
Dabei bezeichnet $b_{\symup{b}}$ die Breite und $h_{\symup{b}}$ die Höhe der Blende.
Die Breite der effektiven Detektorfläche wird mit $b$ bezeichnet und die entsprechende Höhe mit $h$.
Der Abstand von der Folie zur Blende ist durch $c$ gegeben und $d$ beschreibt den Abstand
der Folie und der Detektoroberfläche.
\begin{figure}[h]
  \centering
  \includegraphics[width=\textwidth]{images/skizze.pdf}
  \caption{Skizze zur Berechnung der Größen $b$ und $h$ aus der Geometrie des Versuchsaufbaus.}
  \label{fig:skizze}
\end{figure}

Aus den bekannten Werten \cite{Versuchsanleitung}
\begin{align*}
  b_{\symup{b}} &= \SI{2}{\milli\meter} \,, \\
  h_{\symup{b}} &= \SI{10}{\milli\meter} \,, \\
  c &= \SI{41}{\milli\meter} \,,\\
  d &= \SI{41}{\milli\meter}
\end{align*}
lassen sich die Werte
\begin{align*}
  b &= \SI{2.195}{\milli\meter} \,,\\
  h &= \SI{10.976}{\milli\meter}
\end{align*}
berechnen. Diese charakterisieren die effektive Detektorfläche. Der Raumwinkel kann dann durch
\begin{equation}
  \Delta \Omega = 4 \arctan\left(\frac{b h}{2d \sqrt{4d^2 + b^2 + h^2}}\right)
\end{equation}
berechnet werden \cite{raumwinkel}. Mit den oben genannten Werten ergibt sich
\begin{equation*}
  \Delta \Omega = \SI{0.0118}{sr} \,.
\end{equation*}

Der Raumwinkel um die Quelle herum folgt mit den Werten \cite{Versuchsanleitung}
\begin{align*}
  c &= \SI{97}{\milli\meter} \,, \\
  d &= \SI{101}{\milli\meter}
\end{align*}
zu
\begin{equation*}
  \Delta \Omega_\text{Quelle} = \SI{0.0021}{sr} \,.
\end{equation*}

\newpage
\subsection{Bestimmung der Aktivität der Probe}
\label{subsec:aktivitaet}

Die theoretisch heute zu erwartende Aktivität lässt sich mit Kenntnis der Halbwertszeit
und der Aktivität zu einem früheren Zeitpunkt berechnen. Im Oktober 1994 betrug die
Aktivität der Probe $\SI{330}{\kilo\becquerel}$ \cite{Versuchsanleitung}. Die Halbwertszeit
von 241-Americium beträgt $T_{1/2}=432$ Jahre \cite{t12}. Daraus lässt sich mithilfe von
\begin{align*}
  A_{\text{heute,theo}}&= A_{1994}\exp(-\lambda t) \,, \\
  \lambda&= \frac{\ln(2)}{T_{1/2}}
\end{align*}
die heutige Aktivität bestimmen. Die Zeit beträgt dabei $t=25+1/6$ Jahre. Somit beträgt die
theoretisch zuerwartende heutige Aktivität
\begin{equation*}
  A_{\text{heute,theo}}= \SI{316.94}{\kilo\becquerel} \,.
\end{equation*}

Aus den Messwerten lässt sich die heutige Aktivität unter
Kenntnis des Raumwinkels $\Delta \Omega_{\text{Quelle}}$ aus der Messung der Zählrate berechnen.
Das Verhältnis der heutigen Aktivität und die gemessene Zählrate müssen nämlich in
dem gleichen Verhältnis zueinander stehen, wie der Raumwinkel $\Delta \Omega_{\text{Quelle}}$ und
der gesamte Raumwinkel. Somit ist die heutige Aktivität gegeben durch
\begin{equation*}
  A_{\text{heute,exp}}=\frac{4 \pi  I_0}{\Delta \Omega_{\text{Quelle}}}=\SI{83.80(173)}{\kilo\becquerel} \,.
\end{equation*}
Dabei wurde für die Zählrate ohne Folie $I_0=\SI{15.73(032)}{1 \per\second}$ verwendet.
Der Fehler berechnet sich durch
\begin{equation*}
  \sigma_A=\frac{4 \pi  \sigma_{I_0}}{\Delta \Omega_{\text{Quelle}}} \,.
\end{equation*}
Der experimentell bestimmte Wert weicht damit vom theoretisch bestimmten um
$-73{,}56\%$ ab.

\subsection{Bestimmung des differenziellen Wirkungsquerschnitts}
\label{subsec:wq}
In Tabelle \ref{tab:wq} sind die Messwerte für die Zählrate in Abhängigkeit vom
Winkel aufgeführt. Die Zählzeiten wurden dabei manuell mithilfe einer Stoppuhr bestimmt.
Die Fehler sind gegeben durch $sqrt(N)$, da es sich hier um ein Zählexperiment handelt.
Außerdem befinden sich dort die berechneten differenziellen
Wirkungsquerschnitte. Das Vorgehen zur Berechnung der Wirkungsquerschnitte wird im
Folgenden erläutert.
\begin{table}[htp]
	\begin{center}
    \caption{Messwerte zur Bestimmung des differenziellen Wirkungsquerschnitts, sowie
    die daraus berechneten Aktivitäten und Wirkungsquerschnitte.}
    \label{tab:wq}
		\begin{tabular}{ccccc}
		\toprule
			{$N$} & {$t$/s} & {$I/\mathrm{s^-1}$} & {$\theta$/°} & ${\frac{d\sigma}{d \Omega}}$\\
			\midrule
			1214 \pm 35 & 150  & 8,09 \pm 0,23 &  0,00 & 1,85 \pm 0,07\\
			1271 \pm 36 & 170  & 7,48 \pm 0,21 & -0,50 & 1,71 \pm 0,06\\
			1185 \pm 34 & 180  & 6,58 \pm 0,19 & -1,00 & 1,50 \pm 0,05\\
			1072 \pm 33 & 180  & 5,96 \pm 0,18 & -1,50 & 1,36 \pm 0,05\\
			1109 \pm 33 & 200  & 5,54 \pm 0,17 & -2,00 & 1,27 \pm 0,05\\
			1103 \pm 33 & 200  & 5,51 \pm 0,17 & -2,50 & 1,26 \pm 0,05\\
			930  \pm 31 & 210  & 4,43 \pm 0,15 & -3,00 & 1,01 \pm 0,04\\
			783  \pm 28 & 210  & 3,73 \pm 0,13 & -3,50 & 0,85 \pm 0,04\\
			881  \pm 30 & 210  & 4,20 \pm 0,14 & -4,00 & 0,96 \pm 0,04\\
			854  \pm 29 & 210  & 4,07 \pm 0,14 & -4,50 & 0,93 \pm 0,04\\
			522  \pm 23 & 210  & 2,49 \pm 0,11 & -5,00 & 0,57 \pm 0,03\\
			540  \pm 23 & 270  & 2,00 \pm 0,09 & -5,50 & 0,46 \pm 0,02\\
			533  \pm 23 & 270  & 1,97 \pm 0,09 & -6,00 & 0,45 \pm 0,02\\
			1156 \pm 34 & 150  & 7,71 \pm 0,23  & 0,50 & 1,76 \pm 0,06\\
			1461 \pm 38 & 170  & 8,59 \pm 0,22  & 1,00 & 1,96 \pm 0,07\\
			1269 \pm 36 & 150  & 8,46 \pm 0,24  & 1,50 & 1,93 \pm 0,07\\
			1248 \pm 35 & 150  & 8,32 \pm 0,24  & 2,00 & 1,90 \pm 0,07\\
			1229 \pm 35 & 150  & 8,19 \pm 0,23  & 2,50 & 1,87 \pm 0,07\\
			1328 \pm 36 & 150  & 8,85 \pm 0,24  & 3,00 & 2,02 \pm 0,07\\
			1258 \pm 35 & 150  & 8,39 \pm 0,24  & 3,50 & 1,91 \pm 0,07\\
			1428 \pm 38 & 150  & 9,52 \pm 0,25  & 4,00 & 2,17 \pm 0,07\\
			1251 \pm 35 & 150  & 8,34 \pm 0,24  & 4,50 & 1,90 \pm 0,07\\
			1160 \pm 34 & 150  & 7,73 \pm 0,23  & 5,00 & 1,77 \pm 0,06\\
			1165 \pm 34 & 150  & 7,77 \pm 0,23  & 5,50 & 1,77 \pm 0,06\\
			933  \pm 31 & 150  & 6,22 \pm 0,20  & 6,00 & 1,42 \pm 0,05\\
			811  \pm 28 & 150  & 5,41 \pm 0,19  & 6,50 & 1,23 \pm 0,05\\
			941  \pm 31 & 180  & 5,23 \pm 0,17  & 7,00 & 1,19 \pm 0,05\\
			756  \pm 28 & 180  & 4,20 \pm 0,15  & 7,50 & 0,96 \pm 0,04\\
			762  \pm 28 & 210  & 3,63 \pm 0,13  & 8,00 & 0,83 \pm 0,03\\
			622  \pm 25 & 210  & 2,96 \pm 0,12  & 8,50 & 0,68 \pm 0,03\\
			511  \pm 23 & 210  & 2,43 \pm 0,11  & 9,00 & 0,56 \pm 0,03\\
			502  \pm 22 & 240  & 2,09 \pm 0,09  & 9,50 & 0,48 \pm 0,02\\
			409  \pm 20 & 240  & 1,70 \pm 0,08  & 10,00& 0,39 \pm 0,02\\
		\bottomrule
		\end{tabular}
	\end{center}
\end{table}

Für den Wirkungsquerschnitt gilt \cite{wq}:
\begin{equation}
  \frac{\symup{d}\sigma}{\symup{d}\Omega} = \frac{I}{I_0  N_{\symup{Au}} \Delta x \Delta \Omega} \,.
  \label{eqn:wq}
\end{equation}
Dabei ist $I$ die Zählrate mit Folie, $I_0$ die Zählrate ohne Folie, $ N_{\symup{Au}} \approx \SI{5.91e28}{\per\cubic\metre}$
die Teilchendichte von Gold, $\Delta x$ die Foliendicke und $\Delta \Omega$ der Raumwinkel.
Mit der Zählrate $I_0=\SI{15.73(032)}{1 \per\second}$ ohne Folie, der Schichtdicke $\Delta x =2$\,µm und dem Raumwinkel
aus Kapitel \ref{subsec:raumwinkel} ergeben sich damit die in Tabelle \ref{tab:wq} dargestellten Werte.
Die Fehler werden hier nach der Gauß'schen Fehlerfortpflanzung bestimmt:
\begin{equation*}
  \sigma_{\frac{\symup{d}\sigma}{\symup{d}\Omega}} = \sqrt{\left(\frac{\sigma_N}{t \cdot I_0 \cdot N_{\symup{Au}} \cdot \Delta x \cdot \Delta \Omega}\right)^2
  + \left(\frac{N \cdot \sigma_{I_0}}{t \cdot I_0^2 \cdot N_{\symup{Au}} \cdot \Delta x \cdot \Delta \Omega}\right)^2}
\end{equation*}

Die berechneten differenziellen Wirkungsquerschnitte sind in Abbildung \ref{fig:wq} gegen den Streuwinkel $\theta$ aufgetragen. Außerdem ist
eine zentrierte und eine um $2{,}5$\,° verschobene Theoriekurve zu sehen.
Diese werden gemäß Gleichung \eqref{eqn:rutherford} mit den
Werten $z=2$, $Z=79$ und $E=5{,}5$\,MeV \cite{energie} berechnet.

\begin{figure}
  \centering
  \includegraphics[width=\textwidth]{build/wq.pdf}
  \caption{Berechnete Werte für den differenziellen Wirkungsquerschnitt und Theoriekurve.}
  \label{fig:wq}
\end{figure}

Es ist zu erkennen, dass sich die experimentell bestimmten Werte nicht gut mit den Theoriekurven decken.

\subsection{Untersuchung von Mehrfachstreuung}
\label{subsec:mehrfach}

Für zwei Goldfolien mit den Dicken $\Delta x=2$\,µm und $\Delta x=4$\,µm werden
die Zählraten
\begin{align*}
  I_2&= \SI{15.71(32)}{1\per\second}\,,\\
  I_4&= \SI{17.18(34)}{1\per\second}
\end{align*}
gemessen. Analog zu der Berechnung in Kapitel \ref{subsec:wq} lassen sich die Wirkungsquerschnitte
\begin{align*}
  \left(\frac{d \sigma}{d \Omega}\right)_2&=\SI{3.92(11)e-21}{\metre\squared} \,, \\
  \left(\frac{d \sigma}{d \Omega}\right)_4&=\SI{1.79(05)e-21}{\metre\squared}
\end{align*}
berechnen. Es ist erkennbar, dass der Wirkungsquerschnitt für die dünnere Folie
größer ist.

Da der Winkel bei dieser Messung zu $\theta=0°$ gewählt wurde und die
Rutherford'sche Streuformel \eqref{eqn:rutherford} an dieser Stelle eine Polstelle besitzt,
können die theoretischen Werte hier nicht berechnet werden.

\subsection{Untersuchung der $Z$-Abhängigkeit}
\label{subsec:z}

Zur Untersuchung der Abhängigkeit des differenziellen Wirkungsquerschnittes von der Ordnungszahl $Z$ des Streumaterials werden Folien aus Gold, Platin und
Bismut betrachtet. In Tabelle \ref{tab:elemente} sind die Dicken $\Delta x$ der Folien,
die Ordnungszahlen $Z$ \cite{molmasse}, die Dichten $\rho$ \cite{rho}, die molaren Massen $M$
\cite{molmasse} und die daraus berechneten
Parameter $\frac{I}{N \Delta x}$ aufgeführt. $N$ bezeichnet hier die in Kapitel \ref{subsec:dicke}
bereits eingeführte Teilchendichte des jeweiligen Elements. Der Fehler ergibt sich durch
\begin{equation*}
  \sigma_{\frac{I}{N \Delta x}}=\frac{\sigma_I}{N \Delta x} \,.
\end{equation*}

\begin{table}[htp]
	\begin{center}
    \caption{Messdaten, sowie Daten der Elemente und daraus berechnete Werte.}
    \label{tab:elemente}
		\begin{tabular}{cccccccc}
		\toprule
    {Element}&{$N$}  & {$I/\mathrm{s^{-1}}$} & {$\rho/\frac{\mathrm{g}}{\mathrm{cm}^3}$}
    & {$M/\frac{\mathrm{g}}{\mathrm{Mol}}$} & {$x$/µm} & {$Z$} & {$\frac{I}{Nx}/10^{-23}\frac{\mathrm{m}^2}{\mathrm{s}}$}\\
			\midrule
      Au  &  905 \pm 30 & 1,89 \pm 0,06 & 19,32 & 196,97 & 2 & 79 & 1,60  \pm 0,05\\
      Au  &  836 \pm 29 & 1,74 \pm 0,06 &  19,32 & 196,97 & 4 & 79 & 0,74 \pm 0,03\\
      Pt  &  901 \pm 30 & 1,88 \pm 0,06 &  21,45 & 195,09 & 2 & 78 & 1,42 \pm 0,05\\
      Bi  &  985 \pm 31 & 2,05 \pm 0,07 &   9,80 & 208,98 & 1 & 83 & 7,27 \pm 0,20\\
		\bottomrule
		\end{tabular}
	\end{center}
\end{table}

In Abbildung \ref{fig:z} ist der Parameter $\frac{I}{N \Delta x}$ gegen die Ordnungszahl
aufgetragen. Es sind dabei zwei Werte bei $Z=79$ zu sehen, da zwei unterschiedlich dicke
Goldfolien untersucht wurden. Es ist zu erkennen, dass der Parameter $\frac{I}{N \Delta x}$
tendenziell mit der Ordnungszahl zunimmt. Für $Z=79$ kann wieder die Mehrfachstreuung
beobachtet werden.

\begin{figure}
  \centering
  \includegraphics[width=\textwidth]{build/z.pdf}
  \caption{Auftragung des Parameters $\frac{I}{N \Delta x}$ gegen die Ordnungszahl $Z$.}
  \label{fig:z}
\end{figure}

Theoretisch lässt sich der Parameter $\frac{I}{N \Delta x}$ aus dem
differenziellen Wirkungsquerschnitt bestimmen. Aus Gleichung \eqref{eqn:wq} folgt:
\begin{equation}
  \frac{I}{N \Delta x}= \frac{d \sigma}{d \Omega}I_0 \Delta \Omega \,.
\end{equation}
Da der differenzielle Wirkungsquerschnitt quadratisch in Z ist, ist theoretisch auch ein
quadratischer Zusammenhang zwischen $\frac{I}{N \Delta x}$ und $Z$ zu erwarten.
Es ergeben sich die Theoriewerte
\begin{align*}
  \left(\frac{I}{N \Delta x}\right)_{\text{Au}}&= \SI{5.29(11)e-25}{\metre\squared\per\second} \,, \\
  \left(\frac{I}{N \Delta x}\right)_{\text{Pt}}&= \SI{5.15(11)e-25}{\metre\squared\per\second} \,, \\
  \left(\frac{I}{N \Delta x}\right)_{\text{Bi}}&= \SI{5.83(12)e-25}{\metre\squared\per\second} \,.
\end{align*}
Der Fehler folgt aus der Gauß'schen Fehlerfortpflanzung zu
\begin{equation*}
  \sigma_{\frac{I}{N \Delta x}}=\frac{d \sigma}{d \Omega} \sigma_{I_0} \Delta \Omega \,.
\end{equation*}
Die Theoriewerte liegen allesamt um etwa zwei Größenordnungen unter den experimentell bestimmten.
